\section{Фундаментальные принципы функционального анализа}

\subsection{Принцип равномерной ограниченности}

Пусть $(X, d)$ --- метрическое пространство.

\begin{dfn}{Плотное множество}\\*
  Множество $M$ называется плотным в $X$,
  если $X\subset \overline M$
\end{dfn}

\begin{dfn}{Нигде не плотное множество}\\*
  Множество $M$ нигде не плотно в $X$,
  если внутренность его замыкания
  ${\overline M}^o$ есть пустое множество.

  Эквивалентно,
  множество $M$ нигде не плотно
  в метрическом пространстве $X$,
  если его замыкание
  не содержит открытых шаров
\end{dfn}

\begin{dfn}{Множество первой категории}\\*
  Множество $M$ называется
  множеством первой категории (Бэра),
  если оно представимо в виде
  объединения счётного числа нигде не плотных множеств
  $M = \cup M_n$
\end{dfn}

\begin{dfn}{Множество второй категории}\\*
  Множество $M$ называется
  множеством второй категории,
  если оно не является
  множеством первой категории
\end{dfn}

\begin{thm}{Бэра, о категориях}\\*
  Полное метрическое пространство
  есть множество второй категории.

  Эквивалентно,
  полное метрическое пространство
  не является объединением счётного числа
  нигде нигде не плотных замкнутых множеств.
\end{thm}
\begin{corollary}
  Если полное метрическое пространство
  представимо в виде объединения
  счётного числа замкнутных множеств,
  то хотя бы одно из этих множеств
  содержит непустое открытое множество
\end{corollary}

\begin{thm}{Банаха-Штейнгауса, принцип равномерной ограниченности}\\*
  Пусть
  \begin{align*}
    & X \text{--- банахово}\\
    & Y \text{ --- нормированное}\\
    & (A_\alpha\in\Hom(X,Y); \alpha\in J)\\
    & \forall x\in X \sup_{\alpha\in J} \|A_\alpha x\| < \infty
  \end{align*}
  Тогда
  $$\sup_{\alpha\in J} \|A_\alpha\|
  = \sup_{\alpha\in J; \|x\|=1} \|A_\alpha x\| < \infty$$
\end{thm}
\begin{proof}
  Рассмотрим семейство множеств
  $$X_n = \{x\in X; \sup_{\alpha\in J} \|A_\alpha x\| \leq n\}$$
  Такие множества замкнуты:
  \begin{align*}
    & \left\{
      \begin{aligned}
        & \{x_k; k\in\NN\}\subset X_n\\
        & x_k\to x_0
      \end{aligned}
      \right.\\
      & \implies x_0 \leq n \implies x_0\in X_n
  \end{align*}
  Очевидно, $X = \cup_n X_n$

  Но $X$ --- полное метрическое,
  а значит, по Th.(Baire's category theorem),
  не является объединением счётного числа
  замкнутых нигде не плотных множеств.
  
  Тогда $\exists X_m$,
  содержащее замкнутый шар
  $\overline B(x_0, r) = \{x\in X; \|x-x_0\|\leq r\}$

  Оценим нормы операторов $A_\alpha$
  
  Пусть $\|x\|=1$, тогда:
  \begin{align*}
    & A_\alpha x = A_\alpha \frac{x_0 + rx - x_0}{r}
    = \frac{1}{r} A_\alpha (x_0 + rx) - \frac{1}{r} A_\alpha x_0 \\
    & \text{Но, } \|x_0 + rx - x_0\| = r, \text{поэтому} \\
    & x_0+rx, x_0 \in X_m \\
    & \implies
      \|A_\alpha (x_0+rx)\|,
      \|A_\alpha x_0\| \leq m \\
    & \|A_\alpha x\| \leq \frac{2m}{r} \forall \alpha\in J
  \end{align*}

  Значит,
  $$\sup_{\alpha\in J} \|A_\alpha\| \leq \frac{2m}{r}$$
\end{proof}

\subsection{Теорема Хана-Банаха}

\begin{dfn}{Частичный порядок}\\*
  Говорят, что множество $X$
  частично-упорядоченно,
  если на нём задано бинарное отношение
  $\prec\subset X\times X$, такое что:
  \begin{align*}
    & \forall x\in X & x\prec x \\
    & \forall x,y\in X & (x\prec y)\land (y\prec x) \implies x=y\\
    & \forall x,y,z\in X & (x\prec y)\land (y\prec z) \implies x\prec z
  \end{align*}
\end{dfn}

\begin{dfn}{Максимальный элемент}\\*
  Элемент $x_0$ частично-упорядоченного множества $X$
  называется максимальным элементом, или мажорантой, если
  $$x_0\prec x \implies x_0=x$$
\end{dfn}

\begin{dfn}{Линейный порядок}\\*
  Множество $X$ называется линейно-упорядоченным,
  или \textbf{цепью}, если
  \begin{align*}
    & \text{Оно частично-упорядоченно} \\
    & \forall x_1,x_2\in X & x_1\prec x_2 \lor x_2\prec x_1
  \end{align*}
  То есть, любые два элемента сравнимы
\end{dfn}

\begin{lemma}{Цорна}\\*
  Если частично-упордоченное множество $X$
  таково, что любая цепь в нём имеет верхнюю грань

  То $X$ имеет мажоранту
\end{lemma}

\begin{dfn}{Полунорма}\\*
  Отображение $p : X\mapsto\RR_+$ называется полунормой, если
  \begin{align*}
    & p(0) = 0 \\
    & p(\alpha x) = \left|\alpha\right| p(x) \forall x\in X\\
    & p(x+y) \leq p(x) + p(y) \forall x,y\in X
  \end{align*}
\end{dfn}

\begin{dfn}{Продолжение линейного функционала}\\*
  Пусть $M\subset X$ --- линейное подпространство пространства $X$.
  $f_0: M\mapsto \KK$ --- линейный функционал.

  Линейный функционал $f: X\mapsto\KK$
  называется продолжением линейного функционала $f_0$
  на линейное пространство $X$, если
  $$\forall x\in M \qquad f(x) = f(x_0)$$
\end{dfn}

\begin{thm}{Хана-Банаха, о продолжении линейного функционала с сохранением нормы}\\*
  Пусть
  \begin{align*}
    & X \text{ --- линейное пространство над полем } \KK \\
    & M\subset X \text{ --- линейное подпространство } \\
    & p: M\mapsto\RR_+ \text{ --- полунорма } \\
    & f_0: M\mapsto\KK \text{ --- линейный функционал, такой что}\\
    & \forall x\in M \left|f_0(x)\right| \leq p(x)
  \end{align*}
  Тогда
  \begin{align*}
    & \exists f: X\mapsto\KK \text{ --- линейный функционал, продолжение } f_0\\
    \left\{\begin{aligned}
      & \forall x\in M & f(x) = f_0(x) \\
      & \forall x\in X & \left|f(x)\right| \leq p(x)
    \end{aligned}\right.
  \end{align*}
  То есть, существует продолжение $f_0$
  на всё пространство $X$ с сохранением нормы
\end{thm}
\begin{proof}
\end{proof}

\begin{corollary}{Существование ограниченных линейных функционалов}\\*
  Пусть
  \begin{align*}
    & X \text{ --- непустое линейное нормированное пространство} \\
    & p: x\mapsto \|x\|
  \end{align*}
  Тогда
  \begin{align*}
    & \text{выберем } 0\neq x_0\in X \\
    & M = \{\alpha x_0; \alpha\in\KK\} \\
    & \exists f_0 : M\mapsto\KK \\
    & f_0(x_0) = \|x_0\| \\
    & f_0(\alpha x_0) = \alpha \|x_0\| \\
    & |f_0(\alpha x_0)| = p(\alpha x_0) = |\alpha| \|x_0\| \\
    & \text{То есть, } f_0 \text{ удовлетворяет условиям теоремы } \\
    & \text{Значит, } \exists f\in\Hom(X,\KK) \\
    &
    \left\{\begin{aligned}
      & \forall x\in M & f(x) = f_0(x) \\
      & \forall x\in X & \|f(x)\| \leq p(x) \\
      & \|f\| = 1
    \end{aligned}\right.
  \end{align*}
\end{corollary}
