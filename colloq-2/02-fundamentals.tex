\section{Фундаментальные принципы функционального анализа}

\subsection{Принцип равномерной ограниченности}

Пусть $(X, d)$ --- метрическое пространство.

\begin{dfn}{Плотное множество}
  Множество $M$ называется плотным в $X$,
  если $X\subset \overline M$
\end{dfn}

\begin{dfn}{Нигде не плотное множество}
  Множество $M$ нигде не плотно в $X$,
  если внутренность его замыкания
  ${\overline M}^o$ есть пустое множество.

  Эквивалентно,
  множество $M$ нигде не плотно
  в метрическом пространстве $X$,
  если его замыкание
  не содержит открытых шаров
\end{dfn}

\begin{dfn}{Множество первой категории}
  Множество $M$ называется
  множеством первой категории (Бэра),
  если оно представимо в виде
  объединения счётного числа нигде не плотных множеств
  $M = \cup M_n$
\end{dfn}

\begin{dfn}{Множество второй категории}
  Множество $M$ называется
  множеством второй категории,
  если оно не является
  множеством первой категории
\end{dfn}

\begin{thm}{Бэра, о категориях}
  Полное метрическое пространство
  есть множество второй категории.

  Эквивалентно,
  полное метрическое пространство
  не является объединением счётного числа
  нигде нигде не плотных замкнутых множеств.
\end{thm}
\begin{corollary}
  Если полное метрическое пространство
  представимо в виде объединения
  счётного числа замкнутных множеств,
  то хотя бы одно из этих множеств
  содержит непустое открытое множество
\end{corollary}

\begin{thm}{Банаха-Штейнгауса, принцип равномерной ограниченности}
  
\end{thm}
