\section{Фундаментальные принципы функционального анализа}

\subsection{Принцип равномерной ограниченности}

Пусть $(X, d)$ --- метрическое пространство.

\begin{dfn}{Плотное множество}\\*
  Множество $M$ называется плотным в $X$,
  если $X\subset \overline M$
\end{dfn}

\begin{dfn}{Нигде не плотное множество}\\*
  Множество $M$ нигде не плотно в $X$,
  если внутренность его замыкания
  ${\overline M}^o$ есть пустое множество.

  Эквивалентно,
  множество $M$ нигде не плотно
  в метрическом пространстве $X$,
  если его замыкание
  не содержит открытых шаров
\end{dfn}

\begin{dfn}{Множество первой категории}\\*
  Множество $M$ называется
  множеством первой категории (Бэра),
  если оно представимо в виде
  объединения счётного числа нигде не плотных множеств
  $M = \cup M_n$
\end{dfn}

\begin{dfn}{Множество второй категории}\\*
  Множество $M$ называется
  множеством второй категории,
  если оно не является
  множеством первой категории
\end{dfn}

\begin{thm}{Бэра, о категориях}\\*
  Полное метрическое пространство
  есть множество второй категории.

  Эквивалентно,
  полное метрическое пространство
  не является объединением счётного числа
  нигде нигде не плотных замкнутых множеств.
\end{thm}
\begin{corollary}
  Если полное метрическое пространство
  представимо в виде объединения
  счётного числа замкнутных множеств,
  то хотя бы одно из этих множеств
  содержит непустое открытое множество
\end{corollary}

\begin{thm}{Банаха-Штейнгауса, принцип равномерной ограниченности}\\*
  Пусть
  \begin{align*}
    & X \text{--- банахово}\\
    & Y \text{ --- нормированное}\\
    & (A_\alpha\in\Hom(X,Y); \alpha\in J)\\
    & \forall x\in X \sup_{\alpha\in J} \|A_\alpha x\| < \infty
  \end{align*}
  Тогда
  $$\sup_{\alpha\in J} \|A_\alpha\|
  = \sup_{\alpha\in J; \|x\|=1} \|A_\alpha x\| < \infty$$
\end{thm}
\begin{proof}
  Рассмотрим семейство множеств
  $$X_n = \{x\in X; \sup_{\alpha\in J} \|A_\alpha x\| \leq n\}$$
  Такие множества замкнуты:
  \begin{align*}
    & \left\{
      \begin{aligned}
        & \{x_k; k\in\NN\}\subset X_n\\
        & x_k\to x_0
      \end{aligned}
      \right.\\
      & \implies x_0 \leq n \implies x_0\in X_n
  \end{align*}
  Очевидно, $X = \cup_n X_n$

  Но $X$ --- полное метрическое,
  а значит, по Th.(Baire's category theorem),
  не является объединением счётного числа
  замкнутых нигде не плотных множеств.
  
  Тогда $\exists X_m$,
  содержащее замкнутый шар
  $\overline B(x_0, r) = \{x\in X; \|x-x_0\|\leq r\}$

  Оценим нормы операторов $A_\alpha$
  
  Пусть $\|x\|=1$, тогда:
  \begin{align*}
    & A_\alpha x = A_\alpha \frac{x_0 + rx - x_0}{r}
    = \frac{1}{r} A_\alpha (x_0 + rx) - \frac{1}{r} A_\alpha x_0 \\
    & \text{Но, } \|x_0 + rx - x_0\| = r, \text{поэтому} \\
    & x_0+rx, x_0 \in X_m \\
    & \implies
      \|A_\alpha (x_0+rx)\|,
      \|A_\alpha x_0\| \leq m \\
    & \|A_\alpha x\| \leq \frac{2m}{r} \forall \alpha\in J
  \end{align*}

  Значит,
  $$\sup_{\alpha\in J} \|A_\alpha\| \leq \frac{2m}{r}$$
\end{proof}
