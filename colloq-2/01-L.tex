\section{Ограниченные линейные операторы}

Пусть $X$, $Y$ --- банаховы,
$A : X \mapsto Y$ --- линейный оператор.

\begin{thm}
\label{th:l-b-def-equiv}
Следующие три утверждения эквивалентны:

\begin{equation} \label{eq:lb-1}
  A \text{ непрерывен в каждой точке } x\in X \end{equation}
\begin{equation} \label{eq:lb-2}
  A \text{ непрерывен в нуле } 0\in X \end{equation}
\begin{equation} \label{eq:lb-3}
  A \text{ ограничен в единичном шаре } \{x\in X ; \|x\|\leq 1\end{equation}
\end{thm}
\begin{proof}

1. Очевидно, \eqref{eq:lb-1} $\implies$ \eqref{eq:lb-2} \\

2. \eqref{eq:lb-2} $\implies$ \eqref{eq:lb-3} 

\begin{align*}
& A \text{ непрерывен в } 0\in X \\
& A^{-1}\{y\in Y: \|y\|\leq 1\} = V\text{ есть окрестность нуля в } X \\
& \exists r>0 \quad B(0, r)\subset V \\
& \forall x\in X \quad (\|x\| \leq r) \implies (\|A x\|_Y \leq 1)
\end{align*}

Пусть $\|x\|\leq 1$, тогда $\|rx\| = r\|x\| \leq r$,
следовательно ${\|A rx\| = r \|A x\| \leq 1}$,
следовательно ${\|A x\| \leq \frac{1}{r}}$ для всех $x\in X$, для которых $\|x\|\leq 1$

Это и означает, что $A$ ограничен в единичном шаре.

3. \eqref{eq:lb-3} $\implies$ \eqref{eq:lb-1}:

Пусть
 $$\exists M>0 \quad \forall x\in X \quad \|x\|\leq 1 \implies \|A x\| \leq M$$

Если $x=0$, то $$\|A x\| = 0 \leq M\|x\| = 0$$

Если $x\neq 0$, то
$$\|A x\| = \|A \frac{\|x\|x}{\|x\|}\| = \|x\| \|A \frac{x}{\|x\|}\| \leq M \|x\|$$
поскольку  $\|\frac{x}{\|x\|}\|\leq 1$

То есть $\forall x\in X \quad \|A x\| \leq M \|x\|$.

Покажем теперь, что $\forall x_0\in X$ $A$ непрерывен в $x_0$:
\begin{align*}
& \forall x\in X \text{ из } (\|x - x_0\| \leq \frac{\epsilon}{M}) \text{ следует } \\
& \|A x - A x_0\| = \|A (x - x_0)\| \leq M \frac{\epsilon}{M} = \epsilon
\end{align*}

\end{proof}

Эта теорема даёт основание для следующего определения:

\begin{dfn}{Ограниченный линейный оператор}
  Линейный оператор $A$
  называется ограниченным (или непрерывным) линейным оператором,
  если конечна величина
  $$\|A\| = \sup_{\|x\|\leq 1} A x$$

  Эта величина является нормой:

  \begin{align*}
    & \|0\| = 0 \\
    & \|A\| = 0 \implies 0\leq sup_{\|x\|\leq 1} A x = 0 \implies A = 0 \\
    & \|A + B\| = \sup_{\|x\|\leq 1} (A + B) x \leq \sup_{\|x\|\leq 1} A x + \sup_{\|x\|\leq 1} B x = \|A\| + \|B\| \\
    & \|\alpha A\| = \sup_{\|x\|\leq 1} \alpha A x = \alpha \sup_{\|x\|\leq 1} A x = \alpha \|A x\|
  \end{align*}
\end{dfn}

\begin{thm}{Пространство ограниченных линейных операторов}
  Ограниченные линейные операторы, с определённой выше нормой,
  образуют банахово пространство $\Hom(X,Y)$
\end{thm}
\begin{proof}
\end{proof}
