\section{Ограниченные линейные операторы}

Пусть $X$, $Y$ --- банаховы.

\begin{dfn}{Линейный оператор}
  Отображение $A : X \mapsto Y$ называется линейным оператором, если
  \begin{align*}
    & A (x_1 + x_2) = A x_1 + A x_2 \\
    & A \alpha x = \alpha A x
  \end{align*}
\end{dfn}

\begin{thm}
\label{th:l-b-def-equiv}
Пусть $X$, $Y$ --- банаховы пространства над одним полем $\KK$,
$A : X \mapsto Y$ --- линейный оператор.

Тогда следующие три утверждения эквивалентны:

\begin{equation} \label{eq:lb-1}
  A \text{ непрерывен в каждой точке } x\in X \end{equation}
\begin{equation} \label{eq:lb-2}
  A \text{ непрерывен в нуле } 0\in X \end{equation}
\begin{equation} \label{eq:lb-3}
  A \text{ ограничен в единичном шаре } \{x\in X ; \|x\|\leq 1\} \end{equation}
\end{thm}
\begin{proof}

1. Очевидно, \eqref{eq:lb-1} $\implies$ \eqref{eq:lb-2} \\

2. \eqref{eq:lb-2} $\implies$ \eqref{eq:lb-3} 

\begin{align*}
& A \text{ непрерывен в } 0\in X \\
& A^{-1}\{y\in Y: \|y\|_Y \leq 1\} = V\text{ есть окрестность нуля в } X \\
& \exists r>0 \quad B(0, r)\subset V \\
& \forall x\in X \quad (\|x\| \leq r) \implies (\|A x\|_Y \leq 1)
\end{align*}

Пусть $\|x\|\leq 1$, тогда $\|rx\| = r\|x\| \leq r$,\\
следовательно ${\|A rx\|_Y = r \|A x\|_Y \leq 1}$,\\
следовательно ${\|A x\|_Y \leq \frac{1}{r}}$
для всех $x\in X$, для которых $\|x\|\leq 1$

Это и означает, что $A$ ограничен в единичном шаре.

3. \eqref{eq:lb-3} $\implies$ \eqref{eq:lb-1}:

Пусть
 $$\exists M>0 \quad \forall x\in X \quad \|x\|\leq 1 \implies \|A x\|_Y \leq M$$

Если $x=0$, то $$\|A x\|_Y = 0 \leq M\|x\| = 0$$

Если $x\neq 0$, то
$$\|A x\|_Y = \|A \frac{\|x\|x}{\|x\|}\|_Y = \|x\| \|A \frac{x}{\|x\|}\|_Y \leq M \|x\|$$
поскольку  $\|\frac{x}{\|x\|}\|\leq 1$

То есть $\forall x\in X \quad \|A x\|_Y \leq M \|x\|$.

Покажем теперь, что $\forall x_0\in X$ $A$ непрерывен в $x_0$:
\begin{align*}
& \forall x\in X\\
& \text{из } (\|x - x_0\| \leq \frac{\varepsilon}{M}) \text{ следует } \\
& \|A x - A x_0\|_Y = \|A (x - x_0)\|_Y \leq M \frac{\varepsilon}{M} = \varepsilon
\end{align*}

\end{proof}

Эта теорема даёт основание для следующего определения:

\begin{dfn}{Ограниченный линейный оператор}
  Линейный оператор $A$
  называется ограниченным (или непрерывным) линейным оператором,
  если конечна величина
  $$\|A\| = \sup_{\|x\|\leq 1} A x$$

  Эта величина является нормой:

  \begin{align*}
    & \|0\| = 0 \\
    & \|A\| = 0 \implies 0\leq sup_{\|x\|\leq 1} A x = 0 \implies A = 0 \\
    & \|A + B\| = \sup_{\|x\|\leq 1} (A + B) x \leq \sup_{\|x\|\leq 1} A x + \sup_{\|x\|\leq 1} B x = \|A\| + \|B\| \\
    & \|\alpha A\| = \sup_{\|x\|\leq 1} \|\alpha A x\| = \left|\alpha\right| \sup_{\|x\|\leq 1} \|A x\| = \left|\alpha\right| \|A\|
  \end{align*}
\end{dfn}

Пространство всех линейных ограниченных (i.e. непрерывных отображений) из $X$ в $Y$
образует векторное подпространство $\Hom(X, Y)$
пространства всех линейный отображений из $X$ в $Y$.
Это пространство нормированное с нормой
$$\|A\| = \sup_{\|x\|\leq 1} \|A x\|$$

\begin{thm}{Банахово пространство линейных непрерывных операторов}
  Если $Y$ --- банахово, то и $\Hom(X, Y)$ --- банахово пространство
\end{thm}
\begin{proof}
  Как установлено выше, $\|.\|_{\Hom(X,Y)}$ действительно является нормой.
  Установим полноту.
  \par\hangpara{1em}{1}
  Пусть $(A_n)$ --- фундаментальная последовательность в $\Hom(X,Y)$:
  \begin{align*}
    \forall\varepsilon>0 \exists N \forall n\\
    n>N \implies \|A_n - A_m\| \leq \varepsilon
  \end{align*}
  Наряду с ней рассмотрим последовательность $(A_n x)$ в $Y$.
  Она также фундаментальна:
  \begin{equation*}
    \|A_n x - A_m x\|_Y = \|(A_n - A_m) x\|_Y \leq \|A_n - A_M\|_Y\|x\|_X < \varepsilon\|x\|
  \end{equation*}
  Но $Y$ --- банахово, и любая последовательность Коши в нём сходится
  \begin{equation*}
    \forall x\in X \qquad \lim A_n x = y\in Y
  \end{equation*}
  Введём отображение
  $A:X\mapsto Y:x\mapsto\lim_{n\to\infty} A_n x$.

  \par\hangpara{1em}{1}
  Покажем, что $A\in\Hom(X<Y)$ и $A_n\to A$ по норме:
  \begin{align*}
    &A (x_1 + x_2) = \lim A_n (x_1 + x_2) = \lim (A_n x_1 + A_n x_2) = \\
    &\text{но } A_n x \text{ cходится } \forall x\in X \text{, поэтому } \\
    &= \lim A_n x_1 + \lim A_n x_2 = A x_1 + A x_2 \\
    &A (\alpha x) = \lim A_n \alpha x = \alpha \lim A_n x = \alpha A x
  \end{align*}
  То есть, $A$ линеен.

  Далее
  \begin{align*}
    & \forall x\in X &\\
    & \forall \varepsilon>0 \exists N \forall n>N \\
    & \|(A - A_n) x\| = \|A x - A_n x\| \leq \varepsilon \\
    & \|A - A_n\| = \sup_{\|x\|\leq 1} \|(A - A_n)x\| \leq \varepsilon \\
  \end{align*}
  То есть, $A_n\to A$ по норме.

  Наконец
  \begin{equation*}
    \exists C>0 \\
    \forall x\in X \qquad \|x\|\leq 1 \implies \\
    \|A x\| \leq
     \underbrace{\|A x - A_n x\|}_{\leq \varepsilon}
     + \underbrace{\|A_n x\|}_{\leq C} \leq \varepsilon+C
  \end{equation*}
  То есть, $A$ ограниченный
\end{proof}
