\section{Элементы нелинейного анализа}

\subsection{Производная}
\begin{dfn}{Касательные отображения}
\end{dfn}

\begin{dfn}{Производная в точке}
\end{dfn}

\begin{dfn}{Производная}
\end{dfn}

\begin{thm}{О единственности производной}
\end{thm}

\begin{dfn}{Ряд Тейлора}
\end{dfn}

\begin{dfn}{Экстремум, локальные максимум и минимум}
\end{dfn}

\begin{dfn}{Квадратичная форма}
\end{dfn}

\begin{dfn}{Равномерно положительно-(отрицательно-)определённая квадратичная форма}
\end{dfn}

\begin{thm}{Ферма, о необходимом условии экстремума}
\end{thm}

\begin{thm}{Достаточные условия экстремума}
\end{thm}

\subsection{ФКП}
